\documentclass[aspectratio=169]{beamer}
\input{preamble.tex}

\renewcommand{\footnoterule}{}
\tikzset{
  invisible/.style={opacity=0},
  visible on/.style={alt={#1{}{ invisible}}},
  alt/.code args={<#1>#2#3}{%
    \alt<#1>{\pgfkeysalso{#2}}{\pgfkeysalso{#3}} 
  },
}


\usepackage{adjustbox}
\setbeamertemplate{footline}[frame number]

\newcommand{\backupbegin}{
  \newcounter{finalframe}
    \setcounter{finalframe}{\value{framenumber}}
}

\newcommand{\backupend}{
  \setcounter{framenumber}{\value{finalframe}}
}

\title{Investigating physical constraints underlying catalysis and their impact on metabolic systems}
\author{Uri Barenholz}
\institute{CRI Research Symposium}
\date{October 11, 2017}
\usepackage[absolute,overlay]{textpos}
\newcommand\urltext{
    \begin{textblock*}{\paperwidth}(0pt,\textheight)
        \raggedright \small \url{https://git.io/vd2xO} \hspace{.5em}
    \end{textblock*}
}
\begin{document}

\frame{
  \titlepage
    \urltext
}

\frame{\frametitle{Once upon a time\ldots}
    \begin{adjustbox}{max totalsize={0.6\textwidth}{\textheight},center}
        \pgfimage{car.png}
    \end{adjustbox}
    \urltext
}

% Typical chemical reaction graph
% Two parameters - maximal rate and either selectivity or affinity.
% These parameters are experimentally determined and known for only 5% of enzymes.
% Can we limit these parameters or estimate them theoretically?
% This research investigates potential factors that limit these parameters theoretically.
\frame{\frametitle{Research questions}
\begin{itemize}
    \item What is the physical limit for lowering the activation energy barrier of a given reaction
      \pause
    \item How is the affinity of an enzyme affected by the requirement to be selective
  \end{itemize}
    \urltext
}


%\frame{\frametitle{Is the efficiency of harvesting chemical energy limited?}
%\begin{itemize}
%    \item Given 1 kg of fuel - how much useful energy can we extract?
%    \item What is the expected energy extraction rate/power?
%  \end{itemize}
%    \urltext
%}


%\frame{\frametitle{How is chemical energy stored and harvested?}
%\begin{itemize}
%    \item Chemical energy is stored as a stable local minimal energy state
%    \item To harvest chemical energy, the energy barrier is lowered and the process is coupled to an energy-consuming process
%    \item The rate at which the process proceeds depends on the activation energy barrier
%    \item Enzymes are the biological catalysts that lower the activation energy barrier
%  \end{itemize}
%    \urltext
%}

\frame{\frametitle{Textbook illustration}
    \begin{adjustbox}{max totalsize={0.6\textwidth}{\textheight},center}
        \pgfimage{energyLandscapeLeninger.png}
    \end{adjustbox}
    \urltext
}

\frame{\frametitle{Modeling energy landscape modification in a classical system}
\only<1>{
    \begin{adjustbox}{max totalsize={0.6\textwidth}{\textheight},center}
      \pgfimage{subsbody2.pdf}
    \end{adjustbox}
  }
\only<2>{
    \begin{adjustbox}{max totalsize={0.6\textwidth}{\textheight},center}
      \pgfimage{subsbody3.pdf}
    \end{adjustbox}
  }
 \only<3>{
    \begin{adjustbox}{max totalsize={0.6\textwidth}{\textheight},center}
      \pgfimage{subsbody4.pdf}
    \end{adjustbox}
  }
 \only<4>{
    \begin{adjustbox}{max totalsize={0.6\textwidth}{\textheight},center}
      \pgfimage{subsbody5.pdf}
    \end{adjustbox}
  }
    \urltext
}

\frame{\frametitle{Reaction energy landscape of model substrate}
\begin{columns}
    \begin{column}{0.5\textwidth}
        \pgfimage[width=\textwidth]{subsbody5.pdf}
    \end{column}
    \begin{column}{0.5\textwidth}
        \pgfimage[width=\textwidth]{energylandscapemodificationmodel.pdf}
    \end{column}
\end{columns}
    \urltext
}

\frame{\frametitle{Introducing a model catalyst}
    \begin{adjustbox}{max totalsize={0.6\textwidth}{\textheight},center}
      \only<1>{
        \pgfimage[width=\textwidth]{subscatdist.pdf}
      }
      \only<2>{
        \pgfimage[width=\textwidth]{subscatdistforce.pdf}
      }
     \end{adjustbox}
    \urltext
}

\frame{\frametitle{Reaction energy landscape of bound substrate}
\begin{columns}
    \begin{column}{0.5\textwidth}
        \pgfimage[width=\textwidth]{subscatforces.pdf}
     \end{column}
    \only<2>{
    \begin{column}{0.5\textwidth}
        \pgfimage[width=\textwidth]{energylandscapemodificationmodel2.pdf}
      \end{column}
    }
\end{columns}
    \urltext
}

\frame{\frametitle{The catalyst creates a bypass to the energy barrier at the transition state}
    \begin{adjustbox}{max totalsize={0.6\textwidth}{\textheight},center}
        \pgfimage{catlandscape.pdf}
    \end{adjustbox}
    \urltext
}

\frame{\frametitle{Subtracting the potential field at the transition state from the initial state produces an energy barrier reduction landscape}
    \only<1-2>{
\begin{columns}
    \begin{column}{0.5\textwidth}
      \pgfimage[width=\textwidth]{subsstartpot.pdf}
    \end{column}
    \begin{column}{0.5\textwidth}
    \only<2>{
        \pgfimage[width=\textwidth]{substranspot.pdf}
    }
    \end{column}
\end{columns}
}
    \only<3>{
    \begin{adjustbox}{max totalsize={0.6\textwidth}{\textheight},center}
        \pgfimage{subsdiffpot.pdf}
    \end{adjustbox}
     }
    \urltext
}

\frame{\frametitle{Subtracting the potential field at the transition state from the initial state produces an energy barrier reduction landscape}
\begin{itemize}
    \item The resulting function quantifies the barrier reduction when positioning a positive point charge at any coordinate in space
      \pause
    \item Placing charges at extremum points of this function achieves maximal barrier reduction
  \end{itemize}
    \urltext
}


\frame{\frametitle{Methodological approach for investigating catalytic constraints}
\begin{itemize}
      \pause
     \item Crowd-sourcing platform
      \begin{itemize}
           \item Challenge existing assumptions
           \item Reveal potential catalytic mechanisms
      \end{itemize}
      \pause
     \item Handheld, functional models
      \begin{itemize}
       \item Demonstrate and communicate catalytic principles
      \end{itemize}
      \pause
     \item Apply theoretical framework to molecular domain
      \pause
     \item Investigate metabolic network design implications
      \begin{itemize}
       \item Synthetic biology applications
       \item Origins of life metabolism
      \end{itemize}
  \end{itemize}
    \urltext
}

\frame{\frametitle{Does structural similarity limit affinity in metabolic networks?}
\pause
\vfill
\vfill
    \begin{adjustbox}{max totalsize={0.6\textwidth}{\textheight},center}
        \pgfimage{babyToy.jpg}
    \end{adjustbox}
    \urltext
}

\frame{\frametitle{Does structural similarity limit affinity in metabolic networks?}
\begin{itemize}
  \item Most enzymes are substrate-specific
    \pause
   \item Structural similarity is used for drug discovery and promiscuous activity tests
     \pause
   \item Metabolic networks must contain structurally similar metabolites
    \begin{itemize}
         \item But can potentially reduce similarities at critical points
     \end{itemize}
     \pause
     \item Numerous examples for specificity tradeoffs in the literature
  \end{itemize}
    \urltext
}

\frame{\frametitle{Why do we expect selectivity to decrease affinity?}
    \only<1>{
    \begin{adjustbox}{max totalsize={0.6\textwidth}{\textheight},center}
        \pgfimage{twoshapes.pdf}
      \end{adjustbox}}
      \only<2>{
    \begin{adjustbox}{max totalsize={0.6\textwidth}{\textheight},center}
        \pgfimage{twoshapesslack.pdf}
      \end{adjustbox}}
     \urltext
}


\frame{\frametitle{Examples of specificity-affinity challenges}
\begin{columns}
    \begin{column}{0.5\textwidth}
 \begin{itemize}
    \only<1->{
     \item RuBisCo
      \begin{itemize}
        \item \ce{CO2} versus \ce{O2}
      \end{itemize}
    }
    \only<2->{
     \item Tyrosine ammonia lyase
      \begin{itemize}
        \item \ce{Tyr} versus \ce{Phe}
      \end{itemize}
    }
    \only<3->{
    \item Bacterial DNA methyltransferase
      \begin{itemize}
        \item Relaxing sequence specificity accelerates rate
      \end{itemize}
    \item Bacterial hexose phosphate transporter
    }
       
  \end{itemize}
    \end{column}
    \begin{column}{0.5\textwidth}
    \only<1>{
    \begin{adjustbox}{max totalsize={0.6\textwidth}{\textheight},center}
        \pgfimage{co2o2.jpg}
      \end{adjustbox}}
    \only<2>{
    \begin{adjustbox}{max totalsize={0.6\textwidth}{\textheight},center}
        \pgfimage{tyrosinePhenylalenine.jpg}
      \end{adjustbox}}
      \end{column}
\end{columns}
    \urltext
}

\frame{\frametitle{Can we formulate a quantitative evaluation of the selectivity challenge?}
\begin{itemize}
     \item Given metabolites concentration data
      \begin{itemize}
       \item Identify challenging reactions
       \item Quantify expected cost
      \end{itemize}
      \pause
     \item Given reaction possibilities
      \begin{itemize}
       \item Find biases in metabolic network structure maximizing structural differences
      \end{itemize}
  \end{itemize}
    \urltext
}

\frame{\frametitle{Methodological approach for investigating selectivity tradeoffs}
\begin{itemize}
     \item Impact on metabolites concentrations and enzymes
   \begin{itemize}
       \item BRENDA - identifying weak affinity enzymes
       \item Promiscuous activity data from Sauer lab
       \item Structural similarity metrics comparison with measured metabolites concentrations
   \end{itemize}
   \pause
 \item Impact on network structure
   \begin{itemize}
     \item Project metabolic networks to chemical space
     \item Implement selectivity in constraint based modeling of metabolic networks
   \end{itemize}
 \end{itemize}
    \urltext
}
 
\frame{\frametitle{Summary}
\begin{itemize}
  \item Basic challenges of biological systems are rarely investigated theoretically
     \item Transforming key problems to simplified models in accessible platforms can leverage innovation of wider audience and reveal novel principles
     \item Recently available datasets allow evaluation of hypotheses
     \item Mapping metabolic networks into the chemical space can highlight metabolic network motifs

\end{itemize}
Thank You!
  \urltext
}
% Note battery ineffiency, chemical ~ 20-40 MJ/KG, electrical ~ 0.5 MJ/KG.
% Applications in biosynthesis and degradation.

% Part 2 - selectivity in crowded environments.
% While we commonly acknowledge the importance of selectivity, we do not have a quantitative measure for this challenge.
% Example 1 - selective transport - does thermodynamic favoring towards equal concentrations suffice?
% Example 2 - selectivity vs. affinity - kids match toy - the more configurations we want to accept, the larger the chance we will accept unwanted substrates.

% 
\backupbegin

\frame{\frametitle{Work plan}
    \begin{adjustbox}{max totalsize={\textwidth}{\textheight},center}
        \pgfimage{gantt.pdf}
    \end{adjustbox}
    \urltext
}


\frame{\frametitle{The Michaelis-Menten model for catalyzed chemical reaction rate}
    \begin{adjustbox}{max totalsize={0.6\textwidth}{0.7\textheight},center}
  \begin{tikzpicture}[>=latex',node distance = 2cm]
      \begin{axis}[name=plot1,axis x line=middle,axis y line=left,xlabel near ticks,ylabel near ticks,xmin=0,ymin=-2.5,xmax=2.9,ymax=5.9,xlabel={substrate concentration},ylabel={reaction rate},samples=60,clip=false,yticklabels={,,},xticklabels={,,},tick label style={major tick length=0pt},anchor=left of north west,ylabel style={name=ylabel1}]
        \addplot[domain=0:2.9,autocatacyc,thick] {3*x/(0.1+x)};
        \addplot[domain=0:2.9,gray,dashed] {3};
        \addplot[dashed,gray] coordinates {(0.1,0) (0.1,1.5)};
      \end{axis}
  \end{tikzpicture}
    \end{adjustbox}
   \center{ 
$V=\frac{k_{\text{cat}}[X]}{k_M+[X]}$
}
}

\frame{\frametitle{Catalyst design must track the entire reaction pathway}
    \begin{adjustbox}{max totalsize={0.6\textwidth}{\textheight},center}
        \pgfimage{energylandscapemodificationmodelMax.pdf}
    \end{adjustbox}
    \urltext
}

\frame{\frametitle{References}
\scriptsize{
\begin{enumerate}
  \item Cooper S, et al. (2010) Predicting protein structures with a multiplayer online game. Nature
  \item Cao Y, et al. (2008) ChemmineR: a compound mining framework for R. Bioinformatics 
  \item Reymond J-L (2015) The chemical space project. Acc Chem Res
  \item Wang Y, et al. (2013) fmcsR: mismatch tolerant maximum common substructure searching in R. Bioinformatics
  \item Bar-Even A, et al. (2015) The Moderately Efficient Enzyme: Futile Encounters and Enzyme Floppiness. Biochemistry
  \item Alam MT, et al. (2017) The self-inhibitory nature of metabolic networks and its alleviation through compartmentalization. Nat Commun
  \item Schomburg I, et al. (2004) BRENDA, the enzyme database: updates and major new developments. Nucleic Acids Res
  \item Sévin DC, et al. (2017) Nontargeted in vitro metabolomics for high-throughput identification of novel enzymes in Escherichia coli. Nat Methods
  \item Savir Y, et al. (2010) Cross-species analysis traces adaptation of Rubisco toward optimality in a low-dimensional landscape. PNAS
  \item Tcherkez GGB, et al. (2006) Despite slow catalysis and confused substrate specificity, all ribulose bisphosphate carboxylases may be nearly perfectly optimized. PNAS
  \item Danos V, et al. (2015) Rigid Geometric Constraints for Kappa Models. Electron Notes Theor Comput Sci
\end{enumerate}}
    \urltext
}

\backupend

\end{document}

% Toy model not toy example.
% Winnign strategy of nature is to use enzymes (as opposed to human engineering).
% Cutting edge
% Focus on two question which I find to be essential.
% Then show why they are related. (there is a close connection between these two questions - let me give you an insight how.)
% Toy model needs a lot of work - give intuition - It is well known that the function of any enzyme is intimately connected to the 3d structure of the charge density distribution along the molecule.
% In other words - the position of the charges makes an enzyme work better or worse. it acts on. To 
% Original model.
% Use I.
% Tie to previous experience.
% Say something about personal interdisciplinary experience and relevance to project.
% Add slide with personal experience at the end/beginning, not go over CV, give personal statement - fire.
